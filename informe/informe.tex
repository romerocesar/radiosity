\documentclass[letterpaper]{article}
\usepackage{amsmath,amssymb,amsfonts,textcomp}
\usepackage[spanish]{babel}
\usepackage[utf8]{inputenc}
\usepackage[right=2.5cm, left=2.5cm, top=2.5cm, bottom=2.5cm]{geometry}
\usepackage{enumerate}
\usepackage{nicefrac}
\usepackage{setspace}
\usepackage{url}
\usepackage{listings}
\usepackage{graphicx}

\onehalfspacing

\title{{\bf Radiosity} \\ \large Computación Gráfica II}
\author{Julio C. Castillo 02--34745 \\ César Romero 02--35409}
\date{30 de Enero de 2007}

\begin{document}

%\allowdisplaybreaks
\maketitle
\begin{abstract}
Durante el desarrollo de este proyecto se pretende renderizar la Caja
de Cornell\cite{cornellbox} utilizando nuestra propia implementación
de Radiosity\cite{radiosity}. En este informe se ecuentran detalles de
la implementación que abarcan desde la representación del mundo que
utilizamos hasta la técnica que usamos para hacer el renderizado de la
imagen.
\end{abstract}

\section{Representación del Mundo}
Antes de poder hacer el cómputo necesario para crear una imagen de la
escena que queremos reproducir, es necesario tener claro como se
modelará el mundo sobre el cual se realizarán los cálculos. A
continuación explicamos con detalle nuestra representación tanto en el
archivo de entrada (del cual serán leídos los datos), como en memoria
(sobre los cuales se realizarán los cálculos)

\subsection{El archivo: Formato MDLA}
\label{sec:mdla}
Para la especificación del archivo de entrada se utilizará el formato
MDLA\cite{mdla}, un formato definido conjuntamente por la Universidad
de Cornell y la Universidad de Indiana. Según los autores, el objetivo
de este formato era crear una forma simple, práctica y rápida de
almacenar modelos de escenas.

Se decidió utilizar este formato porque ya existe una especificación
de la Caja de Cornell en este formato. Adicionalmente, está disponible
libremente una librería en C++ que permite leer este tipo de archivos
de manera sencilla.

Es importante destacar que la especificación de la Caja de Cornell
disponible en~\cite{cornellboxdata} define los colores de los
materiales tomando 76 longitudes de onda del espectro visible. En este
proyecto, las longitudes de onda disponibles en la especificación
original serán sustituidas por una tripleta RGB que aproxime al color
en cuestión.

\subsection{Estructura en memoria}
Para representar el ``mundo'' en memoria se utilizara una lista de
\texttt{Shape}s. Cada \texttt{Shape} contiene la toda la informacion
relacionada con una superficie de la escena:
\begin{itemize}
\item Un arreglo de 4 puntos que representan los vertieces de la superficie.
\item Un vector normal a la superficie.
\item El color  o material de la superfice.
\item Un arreglo con los \textsl{patches} que se generan durante la fase de
  particionamiento.
\end{itemize}

\begin{minipage}[c]{0.5\linewidth}
\begin{lstlisting}[language=C++,frame=single,caption=Representación básica del modelo]
typedef float Vector[3]; 
typedef Vector Point;
typedef Vector Color;
typedef Point Quad[4];

struct Shape{
  Point vertices[4];
  Vector normal;
  Color color;
  Quad *patches;

  float area();
};

struct Scene{
  vector<Shape> shape;
};
\end{lstlisting}
\end{minipage}

El hecho de tener un método para calcular el área y un vector normal
al plano que conforma el {\em Shape} facilita la implementación del
particionamiento y el cálculo de los \textsl{form-factors} que serán
explicados con mayor detalle en las secciones \ref{sec:part} y
\ref{sec:ff} respectivamente.

\section{Particionamiento}
\label{sec:part}
Para simplificar la implementación, todos los \textsl{patches} serán
cuadriláteros y serán calculados subdividiendo cada \texttt{Shape}
hasta que cada \textsl{patch} tenga un área máxima que aún está por
ser determinada.

\begin{figure}[htbp]
  \centering
  \input{part.pstex_t}
  \caption{Particionamiento de un cuadrilátero en 2D}
  \label{fig:part}
\end{figure}

El algoritmo de particionamiento que se va a utilizar fue tomado de
\cite{ggems3}, que dado un cuadrilátero, lo divide en cuatro secciones
de igual área tal como se observa en la figura~\ref{fig:part}. La
manera de hacer esto es simplemente promediar las coordenadas de las
esquinas para obtener el punto central y simultaneámente ir calculando
el punto medio entre cada par de vértices. Con esta información es
posible generar los cuatro nuevos cuadriláteros. A continuación se
muestra el pseudocódigo de este algoritmo.
\pagebreak
\begin{lstlisting}[language=pascal,caption=Algoritmo de subdivisión de un cuadrilátero,frame=single,mathescape=True]
procedure splitQuad(q: Quad, var q0,q1,q2,q3: Quad) 
begin
  { calcula el centro y los 4 puntos intermedios entre esquinas }
  center $\leftarrow$  (q[0] + q[1] + q[2] + q[3]) / 4
  mid0 $\leftarrow$ (q[0] + q[1]) / 2
  mid1 $\leftarrow$ (q[1] + q[2]) / 2
  mid2 $\leftarrow$ (q[2] + q[3]) / 2
  mid3 $\leftarrow$ (q[3] + q[0]) / 2

  { construye los 4 nuevos cuadrilateros }
  q0 $\leftarrow$ (q[0], mid0, center, mid3)
  q1 $\leftarrow$ (q[1], mid1, center, mid0)
  q2 $\leftarrow$ (q[2], mid2, center, mid1)
  q3 $\leftarrow$ (q[3], mid3, center, mid2)
end
\end{lstlisting}
 
Las primeras pruebas, sin embargo, serán realizadas utilizando patches
del mismo tamaño que el \texttt{Shape} para poder obtener resultados
rápidos y menos propensos a fallas.

\section{Cómputo de los \textsl{Form-Factor}}
\label{sec:ff}
Para hacer el cómputo de los \textsl{form-factors} se usará un
algoritmo escrito por Filippo Tampieri extraído de~\cite{ggems3}. Este
algoritmo está implementado en C y el cálculo que realiza garantiza
ser muy preciso.

Como se explicó en la sección \ref{sec:part}, para cada subregión del
patch, si esta tiene un área menor que un $\Delta A_r$ definido,
entonces se divide el cuadrilátero en 4 subregiones y se computa
recursivamente el \textsl{form-factor} de cada una. Subdividiendo el
patch receptor en areas cada vez más pequeñas $\Delta A_j$ y
utilizando diferentes aproximaciones constantes de la visibilidad $v$
para cada $\Delta A_j$, la función de visibilidad puede ser aproximada
a cualquier grado de precisión \cite{ggems3}. Así, la contribución de
la fuente $s$ al punto $x$ es
$$\Delta B_r(x)=\rho_rB_s\sum_{j=1}^m(V_jF_{dA_r(x)\rightarrow\Delta A_j}),$$
donde $V_j$ es la fracción de $\Delta A_j$ visible desde $x$.

En la implementación de Tampieri \cite{ggems3} esta función de
visibilidad es externa de manera tal que se adapte a la representación
particular del ``mundo''. El siguiente pseudocódigo muestra el
algoritmo recursivo para computar $\Delta B_r$ que será utilizado en
nuestra implementación

%   $\Delta B_r\leftarrow$ComputeContribution(r(x),s,$\epsilon_B$,$\epsilon_A$);
%   ComputeContribution(r(x),s,$\epsilon_B$,$\epsilon_j$)
%    if CanCull(r(x),s)
%     then return 0;
%     else begin
%      $\epsilon_F\leftarrow\epsilon_B/(\rho_rB_s);$
%      return $\rho_rB_s\cdot$ComputeFormFactor(r(x),s,$\epsilon_F$,$\epsilon_A$);
%      end;

\pagebreak{}
\begin{lstlisting}[language=Pascal,frame=single,caption=Cómputo de form-factors,mathescape=True]

  ComputeFormFactor(r(x),s,$\epsilon_F$,$\epsilon_A$)
   begin 
   V$\leftarrow$ComputeVisibilityFactor(r(x),s);
   if V$\leq$0
    then return 0;
   else if V$\geq$1
    then begin
      $F_{dA_r\rightarrow A_s}\leftarrow$ComputeUnoccludedFormFactor(r(x),s);
      return $F_{dA_r\rightarrow A_s}$;
      end;
   else begin
      $F_{dA_r\rightarrow A_s}\leftarrow$ComputeUnoccludedFormFactor(r(x),s);
      if $F_{dA_r\rightarrow A_s}\leq\epsilon_F\;$or$\;A_s\leq\epsilon_A$
        then return $VF_{dA_r\rightarrow A_s}$;
      else begin
        $(s_1,s_2,...,s_m)\leftarrow$SplitPatch(s);
        return $\sum_{j=1}^m$ComputeFormFactor(r(x),$s_j$,$\epsilon_F$,$\epsilon_A$);
        end;
      end;
   end;
\end{lstlisting}

\subsection{Visibilidad}
\label{sec:vis}

Inicialmente la visibilidad se hará tomando en cuenta los vectores
normales a las superficies que contienen cada uno de estos puntos. El
ángulo que se forma entre estos dos vectores nos da un estimado
inicial de la visibilidad que buscamos.

Otra manera de realizar el cálculo de la visibilidad es usando OpenGL
directamente. Si a cada \textsl{patch} le asignamos un número unívoco
y este lo interpretamos como un color en la imagen, es posible
determinar que \textsl{patches} son visibles consultando el z-buffer
de OpenGL y cambiando el punto de vista del observador. Ésta técnica
es mas robusta con respecto a la oclusión por otras superficies
presentes en la escena. Si bien es más compleja de implementar (por
como se manejan las estructuras de dato) esperamos poder aplicarla en
nuestro proyecto.

% TODO ver que es la vaina de los form-factors precomputados que dijo
% dionisio

\section{Radiosity}
\label{sec:rad}
Para el cómputo de la solución de radiosidad de cada \textsl{patch} de
la escena se utilizará el método de refinamiento progresivo
desrrollado por Cohen \cite{cohen} tal como se describe por Watt \&
Watt \cite{Watt92}. Este método inicia asignando a cada \texttt{patch}
un valor de radiosidad igual a cero o a su valor de
emisividad. Posteriomente se comienza a actualizar las radiocidades de
los patches utilizando el algoritmo que se muestra a continuación.
\pagebreak{}
\begin{lstlisting}[language=pascal,frame=single,mathescape=True]
repeat 
  for each patch $i$
  begin
    Calcular los form-factors de $i$ con el resto de los patches
    for each patch $j \quad (j \neq i)$
    begin
      $\Delta Rad \leftarrow \rho_j \Delta B_i F_{ij}A_i/A_j$
      $\Delta B_j \leftarrow \Delta B_j + \Delta Rad$
      $B_j \leftarrow B_j + \Delta Rad$
    end;   
    $\Delta B_i \leftarrow 0$
  end;
until convergence;
\end{lstlisting}

Cohen llama a esta técina ``\textsl{shooting}'' puesto que la
contribución del \textsl{patch} $i$ es distribuida a todos los demás
\textsl{patches} de la escena\footnote{La técnica tradicional de
  resolución por medio de Gauss-Seidel se conoce como
  ``\textsl{gathering}'' pues se puede interpretar como si calculara
  la contribucion de todos los \textsl{patches} del la escena al
  \textsl{patch} $i$.}. Esta actualización global, permite ir dibujando
la imagen en cada iteración; la idea es que a medida que se va a
avanzando, la calidad y correctitud de la escena va mejorando.

Otro aspecto a considerar del cálculo de radiosidad es que éste debe
realizarce para cada banda de color.  En particular, a mayor número de
bandas de color, mejor resultado se obtiene. Es por esto que la
implementación de la Universidad de Cornell toma en cuenta un total
de 76 bandas de colores. 

Los \textsl{form-factos} también dependen de la longitud de onda, sin
embargo, esta consideración se omite para simplificar la
implementacion. Por lo tantno, para el cálculo de radiosidad en cada
banda se pueden usar los mismo \textsl{form-factors}.

En nuestra implementación tomaremos un total de 3 bandas: rojo, verde
y azul. Las pruebas iniciales se realizarán tomando sólo una banda y
por lo tanto generará una imagen en tonos de grises. Esto es para poder
observar más rapidamente los resultados de la implementación y luego
ampliarla para tomar en cuenta los otras bandas.

\section{Renderizado}
\label{sec:render}
La solución de radiosity nos da un valor constante para todo el
\textsl{patch}. Con un renderer normal, sin embargo, necesitariamos
los valores para cada vértice de la superficie. Estos los podemos
obtener calculando el promedio de las radiosidades de las subregiones
vecinas \cite{Watt92,cohen85}.

La solución de radiosity esta constituida números que toman valores
reales. Estos deben ser escalados a un número entre 0 y 255 para poder
introducirlos en OpenGL. Este escalamiento se debe realizar con una
función sigmoidal \cite{tumblin} porque puede haber valores extremos
que son muy similares cuando se muestran en pantalla. Lo que se quiere
es dar mayor relevancia a valores intermedios.

% TODO buscar cita y funcion sigmoidal.

% TODO ver como es el renderizado usando texturas

\section{Asignación del trabajo}
\label{sec:asig}
El trabajo será dividido en dos grandes partes tal que los componentes
que conforman cada una de estas estén relacionados entre si. 

La primera parte consistirá en leer el archivo de entrada, hacer el
particionamiento y el cómputo de los \textsl{form-factors}. Como
usaremos una librería para la lectura de los datos (como se indica en
la sección \ref{sec:mdla}) y el algoritmo de Tampieri explicado en la
sección \ref{sec:ff}, la complejidad de esta parte es principalmente
la integración del particionamiento al algoritmo y la implementación
de una función de visibilidad que pueda ser usada para el cálculo de
los \textsl{form-factors} sin mayor dificultad.

La segunda parte consistirá en hacer el cálculo de radiosidad una vez
calculados los \textsl{form-factors} y el renderizado de la
escena. Para esta parte es necesario realizar la implementación del
algoritmo de Cohen \cite{cohen} explicado en la sección \ref{sec:rad}
y realizar los cálculos necesarios para la interpolación de los
valores de radiosidad de cada \textsl{patch} y el escalamiento de
dichos valores como se explica en la sección \ref{sec:render}.

No hemos decidido quien estará encargado de cada parte de la
implementación aún.

\bibliographystyle{unsrt}
\bibliography{informe}

\end{document}
